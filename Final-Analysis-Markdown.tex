% Options for packages loaded elsewhere
\PassOptionsToPackage{unicode}{hyperref}
\PassOptionsToPackage{hyphens}{url}
%
\documentclass[
]{article}
\usepackage{amsmath,amssymb}
\usepackage{iftex}
\ifPDFTeX
  \usepackage[T1]{fontenc}
  \usepackage[utf8]{inputenc}
  \usepackage{textcomp} % provide euro and other symbols
\else % if luatex or xetex
  \usepackage{unicode-math} % this also loads fontspec
  \defaultfontfeatures{Scale=MatchLowercase}
  \defaultfontfeatures[\rmfamily]{Ligatures=TeX,Scale=1}
\fi
\usepackage{lmodern}
\ifPDFTeX\else
  % xetex/luatex font selection
\fi
% Use upquote if available, for straight quotes in verbatim environments
\IfFileExists{upquote.sty}{\usepackage{upquote}}{}
\IfFileExists{microtype.sty}{% use microtype if available
  \usepackage[]{microtype}
  \UseMicrotypeSet[protrusion]{basicmath} % disable protrusion for tt fonts
}{}
\makeatletter
\@ifundefined{KOMAClassName}{% if non-KOMA class
  \IfFileExists{parskip.sty}{%
    \usepackage{parskip}
  }{% else
    \setlength{\parindent}{0pt}
    \setlength{\parskip}{6pt plus 2pt minus 1pt}}
}{% if KOMA class
  \KOMAoptions{parskip=half}}
\makeatother
\usepackage{xcolor}
\usepackage[margin=1in]{geometry}
\usepackage{color}
\usepackage{fancyvrb}
\newcommand{\VerbBar}{|}
\newcommand{\VERB}{\Verb[commandchars=\\\{\}]}
\DefineVerbatimEnvironment{Highlighting}{Verbatim}{commandchars=\\\{\}}
% Add ',fontsize=\small' for more characters per line
\usepackage{framed}
\definecolor{shadecolor}{RGB}{248,248,248}
\newenvironment{Shaded}{\begin{snugshade}}{\end{snugshade}}
\newcommand{\AlertTok}[1]{\textcolor[rgb]{0.94,0.16,0.16}{#1}}
\newcommand{\AnnotationTok}[1]{\textcolor[rgb]{0.56,0.35,0.01}{\textbf{\textit{#1}}}}
\newcommand{\AttributeTok}[1]{\textcolor[rgb]{0.13,0.29,0.53}{#1}}
\newcommand{\BaseNTok}[1]{\textcolor[rgb]{0.00,0.00,0.81}{#1}}
\newcommand{\BuiltInTok}[1]{#1}
\newcommand{\CharTok}[1]{\textcolor[rgb]{0.31,0.60,0.02}{#1}}
\newcommand{\CommentTok}[1]{\textcolor[rgb]{0.56,0.35,0.01}{\textit{#1}}}
\newcommand{\CommentVarTok}[1]{\textcolor[rgb]{0.56,0.35,0.01}{\textbf{\textit{#1}}}}
\newcommand{\ConstantTok}[1]{\textcolor[rgb]{0.56,0.35,0.01}{#1}}
\newcommand{\ControlFlowTok}[1]{\textcolor[rgb]{0.13,0.29,0.53}{\textbf{#1}}}
\newcommand{\DataTypeTok}[1]{\textcolor[rgb]{0.13,0.29,0.53}{#1}}
\newcommand{\DecValTok}[1]{\textcolor[rgb]{0.00,0.00,0.81}{#1}}
\newcommand{\DocumentationTok}[1]{\textcolor[rgb]{0.56,0.35,0.01}{\textbf{\textit{#1}}}}
\newcommand{\ErrorTok}[1]{\textcolor[rgb]{0.64,0.00,0.00}{\textbf{#1}}}
\newcommand{\ExtensionTok}[1]{#1}
\newcommand{\FloatTok}[1]{\textcolor[rgb]{0.00,0.00,0.81}{#1}}
\newcommand{\FunctionTok}[1]{\textcolor[rgb]{0.13,0.29,0.53}{\textbf{#1}}}
\newcommand{\ImportTok}[1]{#1}
\newcommand{\InformationTok}[1]{\textcolor[rgb]{0.56,0.35,0.01}{\textbf{\textit{#1}}}}
\newcommand{\KeywordTok}[1]{\textcolor[rgb]{0.13,0.29,0.53}{\textbf{#1}}}
\newcommand{\NormalTok}[1]{#1}
\newcommand{\OperatorTok}[1]{\textcolor[rgb]{0.81,0.36,0.00}{\textbf{#1}}}
\newcommand{\OtherTok}[1]{\textcolor[rgb]{0.56,0.35,0.01}{#1}}
\newcommand{\PreprocessorTok}[1]{\textcolor[rgb]{0.56,0.35,0.01}{\textit{#1}}}
\newcommand{\RegionMarkerTok}[1]{#1}
\newcommand{\SpecialCharTok}[1]{\textcolor[rgb]{0.81,0.36,0.00}{\textbf{#1}}}
\newcommand{\SpecialStringTok}[1]{\textcolor[rgb]{0.31,0.60,0.02}{#1}}
\newcommand{\StringTok}[1]{\textcolor[rgb]{0.31,0.60,0.02}{#1}}
\newcommand{\VariableTok}[1]{\textcolor[rgb]{0.00,0.00,0.00}{#1}}
\newcommand{\VerbatimStringTok}[1]{\textcolor[rgb]{0.31,0.60,0.02}{#1}}
\newcommand{\WarningTok}[1]{\textcolor[rgb]{0.56,0.35,0.01}{\textbf{\textit{#1}}}}
\usepackage{graphicx}
\makeatletter
\def\maxwidth{\ifdim\Gin@nat@width>\linewidth\linewidth\else\Gin@nat@width\fi}
\def\maxheight{\ifdim\Gin@nat@height>\textheight\textheight\else\Gin@nat@height\fi}
\makeatother
% Scale images if necessary, so that they will not overflow the page
% margins by default, and it is still possible to overwrite the defaults
% using explicit options in \includegraphics[width, height, ...]{}
\setkeys{Gin}{width=\maxwidth,height=\maxheight,keepaspectratio}
% Set default figure placement to htbp
\makeatletter
\def\fps@figure{htbp}
\makeatother
\setlength{\emergencystretch}{3em} % prevent overfull lines
\providecommand{\tightlist}{%
  \setlength{\itemsep}{0pt}\setlength{\parskip}{0pt}}
\setcounter{secnumdepth}{-\maxdimen} % remove section numbering
\ifLuaTeX
  \usepackage{selnolig}  % disable illegal ligatures
\fi
\usepackage{bookmark}
\IfFileExists{xurl.sty}{\usepackage{xurl}}{} % add URL line breaks if available
\urlstyle{same}
\hypersetup{
  pdftitle={Google Data Analytics Capstone},
  pdfauthor={Abigail Cabadin},
  hidelinks,
  pdfcreator={LaTeX via pandoc}}

\title{Google Data Analytics Capstone}
\author{Abigail Cabadin}
\date{2024-10-31}

\begin{document}
\maketitle

\subsection{INTRODUCTION}\label{introduction}

Welcome to the Cyclistic bike-share analysis case study! In this case
study, I work for a fictional company, Cyclistic, along with some key
team meambers. In order to answer the business questions, follow the
steps of the data analysis process: Ask, Prepare, Process, Analyze,
Share, and Act.

\subsubsection{SCENARIO}\label{scenario}

I am a junior data analyst working on the marketing analyst team at
Cyclistic, a bike-share company in Chicago. The director of marketing
believes the company's future success depends on maximizing the number
of annual memberships. Therefore my team wants to understand how casual
riders and anuual members use Cyclistic bikes differently.

\subsubsection{ASK}\label{ask}

My Role: I am a Junior Data Analyst working on the marketing analyst
team at Cyclistic. Overall Goal: Design marketing strategies aimed at
converting casual riders into annual members. Business Question: How do
annual members and casual riders use Cyclistic bikes differently?

\paragraph{Load Necessary Packages}\label{load-necessary-packages}

\begin{Shaded}
\begin{Highlighting}[]
\FunctionTok{library}\NormalTok{(tidyverse)}
\FunctionTok{library}\NormalTok{(lubridate)}
\FunctionTok{library}\NormalTok{(dplyr)}
\FunctionTok{library}\NormalTok{(hms)}
\FunctionTok{library}\NormalTok{(data.table)}
\FunctionTok{library}\NormalTok{(ggplot2)}
\end{Highlighting}
\end{Shaded}

\paragraph{Set working directory}\label{set-working-directory}

\begin{Shaded}
\begin{Highlighting}[]
\FunctionTok{setwd}\NormalTok{(}\StringTok{"C:/Users/abiga/OneDrive/Documents/SELF{-}STUDY FILES/Data Analytics Files/Google Data Analytics/Case Study/RAW .csv file"}\NormalTok{)}
\end{Highlighting}
\end{Shaded}

\paragraph{Load .csv file for
analysis}\label{load-.csv-file-for-analysis}

I already cleaned and process the data prior to the analysis.

\begin{Shaded}
\begin{Highlighting}[]
\NormalTok{cyclistic\_df\_analysis }\OtherTok{\textless{}{-}}\FunctionTok{fread}\NormalTok{(}\StringTok{"C:/Users/abiga/OneDrive/Documents/SELF{-}STUDY FILES/Data Analytics Files/Google Data Analytics/Case Study/RAW .csv file/cyclistic\_data\_rmarkdown.csv"}\NormalTok{)}
\end{Highlighting}
\end{Shaded}

\subsection{Analysis}\label{analysis}

\subsubsection{Summary Statistics}\label{summary-statistics}

Calculate summary statistics for key variables to get an overview of the
differences.

\begin{Shaded}
\begin{Highlighting}[]
\NormalTok{summary\_stats }\OtherTok{\textless{}{-}}\NormalTok{ cyclistic\_df\_analysis }\SpecialCharTok{\%\textgreater{}\%}
  \FunctionTok{group\_by}\NormalTok{(member\_casual) }\SpecialCharTok{\%\textgreater{}\%}
  \FunctionTok{summarise}\NormalTok{(}
    \AttributeTok{avg\_ride\_duration =} \FunctionTok{mean}\NormalTok{(ride\_duration, }\AttributeTok{na.rm =} \ConstantTok{TRUE}\NormalTok{),}
    \AttributeTok{median\_ride\_duration =} \FunctionTok{median}\NormalTok{(ride\_duration, }\AttributeTok{na.rm =} \ConstantTok{TRUE}\NormalTok{),}
    \AttributeTok{total\_rides =} \FunctionTok{n}\NormalTok{()}
\NormalTok{  )}
\FunctionTok{print}\NormalTok{(summary\_stats)}
\end{Highlighting}
\end{Shaded}

\begin{verbatim}
## # A tibble: 2 x 4
##   member_casual avg_ride_duration median_ride_duration total_rides
##   <chr>                     <dbl>                <dbl>       <int>
## 1 casual                     21.1                12.0      2124568
## 2 member                     12.2                 8.67     3721469
\end{verbatim}

\paragraph{Interpretation:}\label{interpretation}

Casual riders tend to take longer rides on average compared to annual
members. This might indicate that casual riders use the bikes more for
leisure or longer trips. Meanwhile, Annual Members have a higher total
number of rides, suggesting they use the bikes more frequently, likely
for shorter, more routine trips such as commuting. The median ride
duration being lower than the average for both groups suggest that there
are some longer rides that are skewing the average upwards.

\subsubsection{Ride Duration Analysis}\label{ride-duration-analysis}

Compare the ride durations between annual members and casual riders.

\begin{Shaded}
\begin{Highlighting}[]
\NormalTok{avg\_ride\_duration }\OtherTok{\textless{}{-}}\NormalTok{ cyclistic\_df\_analysis }\SpecialCharTok{\%\textgreater{}\%}
  \FunctionTok{group\_by}\NormalTok{(member\_casual) }\SpecialCharTok{\%\textgreater{}\%}
  \FunctionTok{summarise}\NormalTok{(}\AttributeTok{avg\_duration =} \FunctionTok{mean}\NormalTok{(ride\_duration, }\AttributeTok{na.rm =} \ConstantTok{TRUE}\NormalTok{))}

\FunctionTok{ggplot}\NormalTok{(avg\_ride\_duration, }\FunctionTok{aes}\NormalTok{(}\AttributeTok{x =}\NormalTok{ member\_casual, }\AttributeTok{y =}\NormalTok{ avg\_duration, }\AttributeTok{fill =}\NormalTok{ member\_casual)) }\SpecialCharTok{+}
  \FunctionTok{geom\_bar}\NormalTok{(}\AttributeTok{stat =} \StringTok{"identity"}\NormalTok{) }\SpecialCharTok{+}
  \FunctionTok{labs}\NormalTok{(}\AttributeTok{title =} \StringTok{"Average Ride Duration"}\NormalTok{, }\AttributeTok{x =} \StringTok{"User Type"}\NormalTok{, }\AttributeTok{y =} \StringTok{"Average Duration (minutes)"}\NormalTok{) }\SpecialCharTok{+}
  \FunctionTok{theme\_minimal}\NormalTok{()}
\end{Highlighting}
\end{Shaded}

\includegraphics{Final-Analysis-Markdown_files/figure-latex/unnamed-chunk-4-1.pdf}

\paragraph{Interpretation:}\label{interpretation-1}

The taller bar for casual riders indicates that they generally have
longer ride durations. This could be due to casual riders using the
bikes for leisure activities, sightseeing, or longer trips. While Annual
Members has the shorter bar which suggests that their rides are
typically shorter, likely because they use the bikes for commuting or
quick errands.

\subsubsection{Usage by day of the week}\label{usage-by-day-of-the-week}

\begin{Shaded}
\begin{Highlighting}[]
\NormalTok{usage\_by\_day }\OtherTok{\textless{}{-}}\NormalTok{ cyclistic\_df\_analysis }\SpecialCharTok{\%\textgreater{}\%}
  \FunctionTok{group\_by}\NormalTok{(day\_of\_week, member\_casual) }\SpecialCharTok{\%\textgreater{}\%}
  \FunctionTok{summarise}\NormalTok{(}\AttributeTok{total\_rides =} \FunctionTok{n}\NormalTok{())}
\end{Highlighting}
\end{Shaded}

\begin{verbatim}
## `summarise()` has grouped output by 'day_of_week'. You can override using the
## `.groups` argument.
\end{verbatim}

\begin{Shaded}
\begin{Highlighting}[]
\FunctionTok{ggplot}\NormalTok{(usage\_by\_day, }\FunctionTok{aes}\NormalTok{(}\AttributeTok{x =}\NormalTok{ day\_of\_week, }\AttributeTok{y =}\NormalTok{ total\_rides, }\AttributeTok{fill =}\NormalTok{ member\_casual)) }\SpecialCharTok{+}
  \FunctionTok{geom\_bar}\NormalTok{(}\AttributeTok{stat =} \StringTok{"identity"}\NormalTok{, }\AttributeTok{position =} \StringTok{"dodge"}\NormalTok{) }\SpecialCharTok{+}
  \FunctionTok{labs}\NormalTok{(}\AttributeTok{title =} \StringTok{"Usage by Day of the Week"}\NormalTok{, }\AttributeTok{x =} \StringTok{"Day of the Week"}\NormalTok{, }\AttributeTok{y =} \StringTok{"Total Rides"}\NormalTok{)}
\end{Highlighting}
\end{Shaded}

\includegraphics{Final-Analysis-Markdown_files/figure-latex/unnamed-chunk-5-1.pdf}

\paragraph{Interpretation:}\label{interpretation-2}

Weekend usage: Both casual and members show increased usage on Saturday
and Sunday. Casual riders haver particularly high number of rides on
weekends, indicating leisure or recreational use.

Weekday patterns: Annual members have more consistent usage pattern
throughout the weekdays, with a slight dip on Friday. This suggests that
annual members likely use the bikes for commuting or regular daily
activities. Casual riders have lower usage during weekdays compared to
weekends. This pattern supports the idea that casual riders use the
bikes more for occasional trips rather than daily commuting.

\subsubsection{Hourly Usage Patterns}\label{hourly-usage-patterns}

Examine the hourly usage patterns to see when each group rides the most.

\begin{Shaded}
\begin{Highlighting}[]
\CommentTok{\# Group data by hour and member type}
\NormalTok{hourly\_usage }\OtherTok{\textless{}{-}}\NormalTok{ cyclistic\_df\_analysis }\SpecialCharTok{\%\textgreater{}\%}
  \FunctionTok{group\_by}\NormalTok{(hour, member\_casual) }\SpecialCharTok{\%\textgreater{}\%}
  \FunctionTok{summarise}\NormalTok{(}\AttributeTok{total\_rides =} \FunctionTok{n}\NormalTok{(), }\AttributeTok{.groups =} \StringTok{\textquotesingle{}drop\textquotesingle{}}\NormalTok{)}

\CommentTok{\# Plot the hourly usage patterns}
\FunctionTok{ggplot}\NormalTok{(hourly\_usage, }\FunctionTok{aes}\NormalTok{(}\AttributeTok{x =}\NormalTok{ hour, }\AttributeTok{y =}\NormalTok{ total\_rides, }\AttributeTok{color =}\NormalTok{ member\_casual)) }\SpecialCharTok{+}
  \FunctionTok{geom\_line}\NormalTok{(}\AttributeTok{size =} \DecValTok{1}\NormalTok{) }\SpecialCharTok{+}
  \FunctionTok{labs}\NormalTok{(}\AttributeTok{title =} \StringTok{"Hourly Usage Patterns"}\NormalTok{, }\AttributeTok{x =} \StringTok{"Hour of the Day"}\NormalTok{, }\AttributeTok{y =} \StringTok{"Total Rides"}\NormalTok{) }\SpecialCharTok{+}
  \FunctionTok{theme\_minimal}\NormalTok{() }\SpecialCharTok{+}
  \FunctionTok{scale\_x\_continuous}\NormalTok{(}\AttributeTok{breaks =} \DecValTok{0}\SpecialCharTok{:}\DecValTok{24}\NormalTok{)}
\end{Highlighting}
\end{Shaded}

\begin{verbatim}
## Warning: Using `size` aesthetic for lines was deprecated in ggplot2 3.4.0.
## i Please use `linewidth` instead.
## This warning is displayed once every 8 hours.
## Call `lifecycle::last_lifecycle_warnings()` to see where this warning was
## generated.
\end{verbatim}

\includegraphics{Final-Analysis-Markdown_files/figure-latex/unnamed-chunk-6-1.pdf}

\paragraph{Intepretation:}\label{intepretation}

Morning Peak: Both casual riders and annual members show increased
activity around 7-9 AM, likely corresponding to morning commute times.
Evening Peak: There is another peak around 5-7 PM, which aligns with
evening commute times. Notably, the number of rides for annual members
surpasses casual riders during this period. Midday and Night: Casual
riders have relatively higher usage during midday and late evening
compared to annual members, indicating more leisure or non-commute
trips.

\subsubsection{Usage by Time of Day}\label{usage-by-time-of-day}

\begin{Shaded}
\begin{Highlighting}[]
\CommentTok{\# Group data by time of day and member type}
\NormalTok{time\_of\_day\_usage }\OtherTok{\textless{}{-}}\NormalTok{ cyclistic\_df\_analysis }\SpecialCharTok{\%\textgreater{}\%}
  \FunctionTok{group\_by}\NormalTok{(time\_of\_day, member\_casual) }\SpecialCharTok{\%\textgreater{}\%}
  \FunctionTok{summarise}\NormalTok{(}\AttributeTok{total\_rides =} \FunctionTok{n}\NormalTok{(), }\AttributeTok{.groups =} \StringTok{\textquotesingle{}drop\textquotesingle{}}\NormalTok{)}

\CommentTok{\# Plot the usage patterns by time of day}
\FunctionTok{ggplot}\NormalTok{(time\_of\_day\_usage, }\FunctionTok{aes}\NormalTok{(}\AttributeTok{x =}\NormalTok{ time\_of\_day, }\AttributeTok{y =}\NormalTok{ total\_rides, }\AttributeTok{fill =}\NormalTok{ member\_casual)) }\SpecialCharTok{+}
  \FunctionTok{geom\_bar}\NormalTok{(}\AttributeTok{stat =} \StringTok{"identity"}\NormalTok{, }\AttributeTok{position =} \StringTok{"dodge"}\NormalTok{) }\SpecialCharTok{+}
  \FunctionTok{labs}\NormalTok{(}\AttributeTok{title =} \StringTok{"Usage by Time of Day"}\NormalTok{, }\AttributeTok{x =} \StringTok{"Time of Day"}\NormalTok{, }\AttributeTok{y =} \StringTok{"Total Rides"}\NormalTok{) }\SpecialCharTok{+}
  \FunctionTok{theme\_minimal}\NormalTok{()}
\end{Highlighting}
\end{Shaded}

\includegraphics{Final-Analysis-Markdown_files/figure-latex/unnamed-chunk-7-1.pdf}
\#\#\#\# Interpretation: Afternoon: Casual Riders: The highest number of
rides occurs in the afternoon, indicating that casual riders prefer this
time for their trips. Annual Members: Also show high usage in the
afternoon, but the number of rides is lower compared to casual riders.
Evening: Both casual riders and annual members have significant usage in
the evening, with annual members showing a slightly higher number of
rides. Morning: Usage is lower in the morning for both groups, but
annual members have more rides compared to casual riders, likely due to
commuting. Night: Nighttime usage is the lowest for both groups, with
casual riders having slightly more rides than annual members.

\subsubsection{Most Popular Start
Stations}\label{most-popular-start-stations}

Calculate the most popular start stations for both casual and members.

\begin{Shaded}
\begin{Highlighting}[]
\CommentTok{\# Most popular start stations for casual riders}
\NormalTok{popular\_start\_stations\_casual }\OtherTok{\textless{}{-}}\NormalTok{ cyclistic\_df\_analysis }\SpecialCharTok{\%\textgreater{}\%}
  \FunctionTok{filter}\NormalTok{(member\_casual }\SpecialCharTok{==} \StringTok{"casual"}\NormalTok{) }\SpecialCharTok{\%\textgreater{}\%}
  \FunctionTok{count}\NormalTok{(start\_station\_name, }\AttributeTok{sort =} \ConstantTok{TRUE}\NormalTok{) }\SpecialCharTok{\%\textgreater{}\%}
  \FunctionTok{top\_n}\NormalTok{(}\DecValTok{10}\NormalTok{)}
\end{Highlighting}
\end{Shaded}

\begin{verbatim}
## Selecting by n
\end{verbatim}

\begin{Shaded}
\begin{Highlighting}[]
\CommentTok{\# Most popular start stations for annual members}
\NormalTok{popular\_start\_stations\_members }\OtherTok{\textless{}{-}}\NormalTok{ cyclistic\_df\_analysis }\SpecialCharTok{\%\textgreater{}\%}
  \FunctionTok{filter}\NormalTok{(member\_casual }\SpecialCharTok{==} \StringTok{"member"}\NormalTok{) }\SpecialCharTok{\%\textgreater{}\%}
  \FunctionTok{count}\NormalTok{(start\_station\_name, }\AttributeTok{sort =} \ConstantTok{TRUE}\NormalTok{) }\SpecialCharTok{\%\textgreater{}\%}
  \FunctionTok{top\_n}\NormalTok{(}\DecValTok{10}\NormalTok{)}
\end{Highlighting}
\end{Shaded}

\begin{verbatim}
## Selecting by n
\end{verbatim}

\begin{Shaded}
\begin{Highlighting}[]
\FunctionTok{print}\NormalTok{(popular\_start\_stations\_casual)}
\end{Highlighting}
\end{Shaded}

\begin{verbatim}
##                     start_station_name      n
##                                 <char>  <int>
##  1:                                    406958
##  2:            Streeter Dr & Grand Ave  48972
##  3:  DuSable Lake Shore Dr & Monroe St  32779
##  4:              Michigan Ave & Oak St  24617
##  5: DuSable Lake Shore Dr & North Blvd  22634
##  6:                    Millennium Park  21702
##  7:                     Shedd Aquarium  20470
##  8:                     Dusable Harbor  17950
##  9:                Theater on the Lake  16390
## 10:              Michigan Ave & 8th St  13197
\end{verbatim}

\begin{Shaded}
\begin{Highlighting}[]
\FunctionTok{print}\NormalTok{(popular\_start\_stations\_members)}
\end{Highlighting}
\end{Shaded}

\begin{verbatim}
##               start_station_name      n
##                           <char>  <int>
##  1:                              649099
##  2:     Kingsbury St & Kinzie St  28708
##  3: Clinton St & Washington Blvd  28085
##  4:            Clark St & Elm St  24565
##  5:      Clinton St & Madison St  24461
##  6:        Wells St & Concord Ln  20759
##  7:            Wells St & Elm St  20400
##  8:    Clinton St & Jackson Blvd  19480
##  9:       State St & Chicago Ave  18973
## 10:        Dearborn St & Erie St  18966
\end{verbatim}

\subsubsection{Most Popular End
Stations}\label{most-popular-end-stations}

Similarly, calculate the most popular end stations.

\begin{Shaded}
\begin{Highlighting}[]
\CommentTok{\# Most popular end stations for casual riders}
\NormalTok{popular\_end\_stations\_casual }\OtherTok{\textless{}{-}}\NormalTok{ cyclistic\_df\_analysis }\SpecialCharTok{\%\textgreater{}\%}
  \FunctionTok{filter}\NormalTok{(member\_casual }\SpecialCharTok{==} \StringTok{"casual"}\NormalTok{) }\SpecialCharTok{\%\textgreater{}\%}
  \FunctionTok{count}\NormalTok{(end\_station\_name, }\AttributeTok{sort =} \ConstantTok{TRUE}\NormalTok{) }\SpecialCharTok{\%\textgreater{}\%}
  \FunctionTok{top\_n}\NormalTok{(}\DecValTok{10}\NormalTok{)}
\end{Highlighting}
\end{Shaded}

\begin{verbatim}
## Selecting by n
\end{verbatim}

\begin{Shaded}
\begin{Highlighting}[]
\CommentTok{\# Most popular end stations for annual members}
\NormalTok{popular\_end\_stations\_members }\OtherTok{\textless{}{-}}\NormalTok{ cyclistic\_df\_analysis }\SpecialCharTok{\%\textgreater{}\%}
  \FunctionTok{filter}\NormalTok{(member\_casual }\SpecialCharTok{==} \StringTok{"member"}\NormalTok{) }\SpecialCharTok{\%\textgreater{}\%}
  \FunctionTok{count}\NormalTok{(end\_station\_name, }\AttributeTok{sort =} \ConstantTok{TRUE}\NormalTok{) }\SpecialCharTok{\%\textgreater{}\%}
  \FunctionTok{top\_n}\NormalTok{(}\DecValTok{10}\NormalTok{)}
\end{Highlighting}
\end{Shaded}

\begin{verbatim}
## Selecting by n
\end{verbatim}

\begin{Shaded}
\begin{Highlighting}[]
\FunctionTok{print}\NormalTok{(popular\_end\_stations\_casual)}
\end{Highlighting}
\end{Shaded}

\begin{verbatim}
##                       end_station_name      n
##                                 <char>  <int>
##  1:                                    448083
##  2:            Streeter Dr & Grand Ave  52638
##  3:  DuSable Lake Shore Dr & Monroe St  30642
##  4: DuSable Lake Shore Dr & North Blvd  25973
##  5:              Michigan Ave & Oak St  25185
##  6:                    Millennium Park  23503
##  7:                     Shedd Aquarium  18518
##  8:                Theater on the Lake  17633
##  9:                     Dusable Harbor  16162
## 10:              Michigan Ave & 8th St  11973
\end{verbatim}

\begin{Shaded}
\begin{Highlighting}[]
\FunctionTok{print}\NormalTok{(popular\_end\_stations\_members)}
\end{Highlighting}
\end{Shaded}

\begin{verbatim}
##                 end_station_name      n
##                           <char>  <int>
##  1:                              635656
##  2: Clinton St & Washington Blvd  29003
##  3:     Kingsbury St & Kinzie St  28702
##  4:      Clinton St & Madison St  25661
##  5:            Clark St & Elm St  24693
##  6:        Wells St & Concord Ln  20828
##  7:            Wells St & Elm St  20182
##  8:    Clinton St & Jackson Blvd  19654
##  9:       State St & Chicago Ave  19504
## 10:     University Ave & 57th St  19093
\end{verbatim}

\paragraph{Visualize the most popular stations to better understand the
differences.}\label{visualize-the-most-popular-stations-to-better-understand-the-differences.}

\begin{Shaded}
\begin{Highlighting}[]
\CommentTok{\# Visualization for start stations}
\FunctionTok{ggplot}\NormalTok{(popular\_start\_stations\_casual, }\FunctionTok{aes}\NormalTok{(}\AttributeTok{x =} \FunctionTok{reorder}\NormalTok{(start\_station\_name, n), }\AttributeTok{y =}\NormalTok{ n, }\AttributeTok{fill =} \StringTok{"casual"}\NormalTok{)) }\SpecialCharTok{+}
  \FunctionTok{geom\_bar}\NormalTok{(}\AttributeTok{stat =} \StringTok{"identity"}\NormalTok{) }\SpecialCharTok{+}
  \FunctionTok{coord\_flip}\NormalTok{() }\SpecialCharTok{+}
  \FunctionTok{labs}\NormalTok{(}\AttributeTok{title =} \StringTok{"Top 10 Start Stations for Casual Riders"}\NormalTok{, }\AttributeTok{x =} \StringTok{"Start Station"}\NormalTok{, }\AttributeTok{y =} \StringTok{"Number of Rides"}\NormalTok{) }\SpecialCharTok{+}
  \FunctionTok{theme\_minimal}\NormalTok{()}
\end{Highlighting}
\end{Shaded}

\includegraphics{Final-Analysis-Markdown_files/figure-latex/unnamed-chunk-10-1.pdf}

\begin{Shaded}
\begin{Highlighting}[]
\FunctionTok{ggplot}\NormalTok{(popular\_start\_stations\_members, }\FunctionTok{aes}\NormalTok{(}\AttributeTok{x =} \FunctionTok{reorder}\NormalTok{(start\_station\_name, n), }\AttributeTok{y =}\NormalTok{ n, }\AttributeTok{fill =} \StringTok{"member"}\NormalTok{)) }\SpecialCharTok{+}
  \FunctionTok{geom\_bar}\NormalTok{(}\AttributeTok{stat =} \StringTok{"identity"}\NormalTok{) }\SpecialCharTok{+}
  \FunctionTok{coord\_flip}\NormalTok{() }\SpecialCharTok{+}
  \FunctionTok{labs}\NormalTok{(}\AttributeTok{title =} \StringTok{"Top 10 Start Stations for Annual Members"}\NormalTok{, }\AttributeTok{x =} \StringTok{"Start Station"}\NormalTok{, }\AttributeTok{y =} \StringTok{"Number of Rides"}\NormalTok{) }\SpecialCharTok{+}
  \FunctionTok{theme\_minimal}\NormalTok{()}
\end{Highlighting}
\end{Shaded}

\includegraphics{Final-Analysis-Markdown_files/figure-latex/unnamed-chunk-10-2.pdf}
\#\#\# Visualize most popular end stations

\begin{Shaded}
\begin{Highlighting}[]
\CommentTok{\# Most popular end stations for casual riders}
\NormalTok{popular\_end\_stations\_casual }\OtherTok{\textless{}{-}}\NormalTok{ cyclistic\_df\_analysis }\SpecialCharTok{\%\textgreater{}\%}
  \FunctionTok{filter}\NormalTok{(member\_casual }\SpecialCharTok{==} \StringTok{"casual"}\NormalTok{) }\SpecialCharTok{\%\textgreater{}\%}
  \FunctionTok{count}\NormalTok{(end\_station\_name, }\AttributeTok{sort =} \ConstantTok{TRUE}\NormalTok{) }\SpecialCharTok{\%\textgreater{}\%}
  \FunctionTok{top\_n}\NormalTok{(}\DecValTok{10}\NormalTok{)}
\end{Highlighting}
\end{Shaded}

\begin{verbatim}
## Selecting by n
\end{verbatim}

\begin{Shaded}
\begin{Highlighting}[]
\CommentTok{\# Most popular end stations for annual members}
\NormalTok{popular\_end\_stations\_members }\OtherTok{\textless{}{-}}\NormalTok{ cyclistic\_df\_analysis }\SpecialCharTok{\%\textgreater{}\%}
  \FunctionTok{filter}\NormalTok{(member\_casual }\SpecialCharTok{==} \StringTok{"member"}\NormalTok{) }\SpecialCharTok{\%\textgreater{}\%}
  \FunctionTok{count}\NormalTok{(end\_station\_name, }\AttributeTok{sort =} \ConstantTok{TRUE}\NormalTok{) }\SpecialCharTok{\%\textgreater{}\%}
  \FunctionTok{top\_n}\NormalTok{(}\DecValTok{10}\NormalTok{)}
\end{Highlighting}
\end{Shaded}

\begin{verbatim}
## Selecting by n
\end{verbatim}

\begin{Shaded}
\begin{Highlighting}[]
\CommentTok{\# Visualization for end stations for casual riders}
\FunctionTok{ggplot}\NormalTok{(popular\_end\_stations\_casual, }\FunctionTok{aes}\NormalTok{(}\AttributeTok{x =} \FunctionTok{reorder}\NormalTok{(end\_station\_name, n), }\AttributeTok{y =}\NormalTok{ n, }\AttributeTok{fill =} \StringTok{"casual"}\NormalTok{)) }\SpecialCharTok{+}
  \FunctionTok{geom\_bar}\NormalTok{(}\AttributeTok{stat =} \StringTok{"identity"}\NormalTok{) }\SpecialCharTok{+}
  \FunctionTok{coord\_flip}\NormalTok{() }\SpecialCharTok{+}
  \FunctionTok{labs}\NormalTok{(}\AttributeTok{title =} \StringTok{"Top 10 End Stations for Casual Riders"}\NormalTok{, }\AttributeTok{x =} \StringTok{"End Station"}\NormalTok{, }\AttributeTok{y =} \StringTok{"Number of Rides"}\NormalTok{) }\SpecialCharTok{+}
  \FunctionTok{theme\_minimal}\NormalTok{()}
\end{Highlighting}
\end{Shaded}

\includegraphics{Final-Analysis-Markdown_files/figure-latex/unnamed-chunk-11-1.pdf}

\begin{Shaded}
\begin{Highlighting}[]
\CommentTok{\# Visualization for end stations for annual members}
\FunctionTok{ggplot}\NormalTok{(popular\_end\_stations\_members, }\FunctionTok{aes}\NormalTok{(}\AttributeTok{x =} \FunctionTok{reorder}\NormalTok{(end\_station\_name, n), }\AttributeTok{y =}\NormalTok{ n, }\AttributeTok{fill =} \StringTok{"member"}\NormalTok{)) }\SpecialCharTok{+}
  \FunctionTok{geom\_bar}\NormalTok{(}\AttributeTok{stat =} \StringTok{"identity"}\NormalTok{) }\SpecialCharTok{+}
  \FunctionTok{coord\_flip}\NormalTok{() }\SpecialCharTok{+}
  \FunctionTok{labs}\NormalTok{(}\AttributeTok{title =} \StringTok{"Top 10 End Stations for Annual Members"}\NormalTok{, }\AttributeTok{x =} \StringTok{"End Station"}\NormalTok{, }\AttributeTok{y =} \StringTok{"Number of Rides"}\NormalTok{) }\SpecialCharTok{+}
  \FunctionTok{theme\_minimal}\NormalTok{()}
\end{Highlighting}
\end{Shaded}

\includegraphics{Final-Analysis-Markdown_files/figure-latex/unnamed-chunk-11-2.pdf}
\#\#\#\# Interpretation of the Charts 1. Top 10 End Stations for Casual
Riders Popular End Stations: The most popular end stations for casual
riders include DuSable Lake Shore Dr \& Monroe St, DuSable Lake Shore Dr
\& North Blvd, and Michigan Ave \& Oak St. Leisure Spots: These stations
are near popular tourist attractions and leisure spots, indicating that
casual riders often use the bikes for sightseeing and recreational
purposes. 2. Top 10 End Stations for Annual Members Popular End
Stations: The top end stations for annual members are Clinton St \&
Washington Blvd, Kingsbury St \& Kinzie St, and Clinton St \& Madison
St. Commuter Hubs: These stations are located near business districts
and transit hubs, suggesting that annual members primarily use the bikes
for commuting and regular daily activities. 3. Top 10 Start Stations for
Casual Riders Popular Start Stations: The most popular start stations
for casual riders are Streeter Dr \& Grand Ave, DuSable Lake Shore Dr \&
Monroe St, and Michigan Ave \& Oak St. Tourist Areas: Similar to the end
stations, these start stations are also near tourist attractions,
reinforcing the idea that casual riders use the bikes for leisure and
sightseeing. 4. Top 10 Start Stations for Annual Members Popular Start
Stations: The top start stations for annual members include Kingsbury St
\& Kinzie St, Clinton St \& Washington Blvd, and Clinton St \& Madison
St. Residential and Business Areas: These stations are located near
residential areas and business districts, indicating that annual members
use the bikes for commuting from home to work or other regular
destinations.

\paragraph{Relating the Charts}\label{relating-the-charts}

Usage Patterns: Casual Riders: Both the start and end stations for
casual riders are concentrated around tourist and leisure areas. This
suggests that casual riders typically use the bikes for recreational
purposes, starting and ending their rides at popular attractions. Annual
Members: The start and end stations for annual members are located near
business districts and transit hubs, indicating a pattern of commuting.
Annual members likely use the bikes for daily travel between home, work,
and other routine destinations. Strategic Insights: Bike Availability:
Ensure a higher availability of bikes at tourist spots during weekends
and holidays to cater to casual riders. Commuter Support: Maintain a
steady supply of bikes at residential and business areas during weekdays
to support the commuting needs of annual members. Marketing Strategies:
Casual Riders: Promote leisure and sightseeing packages, offering
discounts or special deals for rides starting or ending at popular
tourist spots. Annual Members: Focus on convenience and reliability,
highlighting the benefits of using bikes for daily commutes and offering
membership incentives.

\subsubsection{Seasonal Usage}\label{seasonal-usage}

\begin{Shaded}
\begin{Highlighting}[]
\CommentTok{\# Seasonal usage}
\NormalTok{seasonal\_usage }\OtherTok{\textless{}{-}}\NormalTok{ cyclistic\_df\_analysis }\SpecialCharTok{\%\textgreater{}\%}
  \FunctionTok{group\_by}\NormalTok{(season, member\_casual) }\SpecialCharTok{\%\textgreater{}\%}
  \FunctionTok{summarise}\NormalTok{(}\AttributeTok{total\_rides =} \FunctionTok{n}\NormalTok{())}
\end{Highlighting}
\end{Shaded}

\begin{verbatim}
## `summarise()` has grouped output by 'season'. You can override using the
## `.groups` argument.
\end{verbatim}

\begin{Shaded}
\begin{Highlighting}[]
\FunctionTok{ggplot}\NormalTok{(seasonal\_usage, }\FunctionTok{aes}\NormalTok{(}\AttributeTok{x =}\NormalTok{ season, }\AttributeTok{y =}\NormalTok{ total\_rides, }\AttributeTok{fill =}\NormalTok{ member\_casual)) }\SpecialCharTok{+}
  \FunctionTok{geom\_bar}\NormalTok{(}\AttributeTok{stat =} \StringTok{"identity"}\NormalTok{, }\AttributeTok{position =} \StringTok{"dodge"}\NormalTok{) }\SpecialCharTok{+}
  \FunctionTok{labs}\NormalTok{(}\AttributeTok{title =} \StringTok{"Seasonal Usage"}\NormalTok{, }\AttributeTok{x =} \StringTok{"Season"}\NormalTok{, }\AttributeTok{y =} \StringTok{"Total Rides"}\NormalTok{)}
\end{Highlighting}
\end{Shaded}

\includegraphics{Final-Analysis-Markdown_files/figure-latex/unnamed-chunk-12-1.pdf}

\paragraph{Interpretation:}\label{interpretation-3}

Summer Peak: Both casual riders and annual members have the highest
number of rides in the summer. This indicates that summer is the most
popular season for bike usage, likely due to favorable weather
conditions. Winter Drop: There is a significant drop in the number of
rides during winter for both groups. This suggests that colder weather
and possibly snow or ice reduce bike usage. Consistent Member Usage:
Annual members have higher usage than casual riders in all seasons
except winter. This consistency indicates that annual members use bikes
more regularly throughout the year, likely for commuting. Casual Riders:
Casual riders show a more pronounced seasonal variation, with a sharp
increase in summer and a steep decline in winter. This pattern suggests
that casual riders are more influenced by seasonal weather changes and
use bikes more for leisure. Conclusion The chart highlights the impact
of seasonal changes on bike usage, with summer being the peak season and
winter showing the lowest usage. Annual members exhibit more consistent
usage across seasons, while casual riders are more affected by weather
conditions. This information can help in planning bike availability and
maintenance schedules throughout the year.

\subsubsection{Rideable Type Analysis}\label{rideable-type-analysis}

\paragraph{Count of Rides by Bike Type and User
Type}\label{count-of-rides-by-bike-type-and-user-type}

\begin{Shaded}
\begin{Highlighting}[]
\CommentTok{\# Group data by rideable type and member type}
\NormalTok{rideable\_type\_usage }\OtherTok{\textless{}{-}}\NormalTok{ cyclistic\_df\_analysis }\SpecialCharTok{\%\textgreater{}\%}
  \FunctionTok{group\_by}\NormalTok{(rideable\_type, member\_casual) }\SpecialCharTok{\%\textgreater{}\%}
  \FunctionTok{summarise}\NormalTok{(}\AttributeTok{total\_rides =} \FunctionTok{n}\NormalTok{(), }\AttributeTok{.groups =} \StringTok{\textquotesingle{}drop\textquotesingle{}}\NormalTok{)}

\CommentTok{\# Plot the usage patterns by rideable type}
\FunctionTok{ggplot}\NormalTok{(rideable\_type\_usage, }\FunctionTok{aes}\NormalTok{(}\AttributeTok{x =}\NormalTok{ rideable\_type, }\AttributeTok{y =}\NormalTok{ total\_rides, }\AttributeTok{fill =}\NormalTok{ member\_casual)) }\SpecialCharTok{+}
  \FunctionTok{geom\_bar}\NormalTok{(}\AttributeTok{stat =} \StringTok{"identity"}\NormalTok{, }\AttributeTok{position =} \StringTok{"dodge"}\NormalTok{) }\SpecialCharTok{+}
  \FunctionTok{labs}\NormalTok{(}\AttributeTok{title =} \StringTok{"Usage by Bike Type and User Type"}\NormalTok{, }\AttributeTok{x =} \StringTok{"Bike Type"}\NormalTok{, }\AttributeTok{y =} \StringTok{"Total Rides"}\NormalTok{) }\SpecialCharTok{+}
  \FunctionTok{theme\_minimal}\NormalTok{()}
\end{Highlighting}
\end{Shaded}

\includegraphics{Final-Analysis-Markdown_files/figure-latex/unnamed-chunk-13-1.pdf}
\#\#\#\# Average Ride Duration by Bike Type and User Type

\begin{Shaded}
\begin{Highlighting}[]
\CommentTok{\# Calculate average ride duration by rideable type and member type}
\NormalTok{avg\_ride\_duration }\OtherTok{\textless{}{-}}\NormalTok{ cyclistic\_df\_analysis }\SpecialCharTok{\%\textgreater{}\%}
  \FunctionTok{group\_by}\NormalTok{(rideable\_type, member\_casual) }\SpecialCharTok{\%\textgreater{}\%}
  \FunctionTok{summarise}\NormalTok{(}\AttributeTok{avg\_duration =} \FunctionTok{mean}\NormalTok{(ride\_duration, }\AttributeTok{na.rm =} \ConstantTok{TRUE}\NormalTok{), }\AttributeTok{.groups =} \StringTok{\textquotesingle{}drop\textquotesingle{}}\NormalTok{)}

\CommentTok{\# Plot the average ride duration by rideable type}
\FunctionTok{ggplot}\NormalTok{(avg\_ride\_duration, }\FunctionTok{aes}\NormalTok{(}\AttributeTok{x =}\NormalTok{ rideable\_type, }\AttributeTok{y =}\NormalTok{ avg\_duration, }\AttributeTok{fill =}\NormalTok{ member\_casual)) }\SpecialCharTok{+}
  \FunctionTok{geom\_bar}\NormalTok{(}\AttributeTok{stat =} \StringTok{"identity"}\NormalTok{, }\AttributeTok{position =} \StringTok{"dodge"}\NormalTok{) }\SpecialCharTok{+}
  \FunctionTok{labs}\NormalTok{(}\AttributeTok{title =} \StringTok{"Average Ride Duration by Bike Type and User Type"}\NormalTok{, }\AttributeTok{x =} \StringTok{"Bike Type"}\NormalTok{, }\AttributeTok{y =} \StringTok{"Average Duration (minutes)"}\NormalTok{) }\SpecialCharTok{+}
  \FunctionTok{theme\_minimal}\NormalTok{()}
\end{Highlighting}
\end{Shaded}

\includegraphics{Final-Analysis-Markdown_files/figure-latex/unnamed-chunk-14-1.pdf}
\#\#\#\# Interpretation of the Graphs 1. Usage by Bike Type and User
Type Classic Bikes: Members: Significantly higher usage compared to
casual riders. Casual Riders: Lower usage but still substantial.
Electric Bikes: Members: High usage, though slightly less than classic
bikes. Casual Riders: Moderate usage, higher than classic bikes.
Electric Scooters: Both Groups: Very low usage compared to bikes,
indicating less popularity. 2. Average Ride Duration by Bike Type and
User Type Classic Bikes: Casual Riders: Longest average ride duration,
around 25 minutes. Members: Shorter average duration, around 15 minutes.
Electric Bikes: Casual Riders: Average duration around 20 minutes.
Members: Shorter average duration, around 12 minutes. Electric Scooters:
Casual Riders: Average duration around 15 minutes. Members: Shortest
average duration, around 10 minutes. Relating the Graphs Usage Patterns:
Members prefer classic and electric bikes for frequent, shorter rides,
likely for commuting. Casual riders use all bike types for longer, less
frequent rides, indicating leisure use. Operational Insights: Ensure a
higher availability of classic and electric bikes for members. Maintain
a balanced supply of all bike types for casual riders, especially in
tourist areas. Marketing Strategies: Promote membership benefits for
frequent commuters. Offer special deals on longer rides for casual
users, highlighting leisure and exploration opportunities.

\subsection{Recommendations}\label{recommendations}

Based on the data and interpretations from the charts, here are some
targeted recommendations for the marketing team to convert casual riders
to annual members:

\subsubsection{Promote Membership Benefits During Peak
Seasons}\label{promote-membership-benefits-during-peak-seasons}

Summer Campaigns: Since both casual and annual members have the highest
usage in summer, launch aggressive marketing campaigns during this
season. Highlight the benefits of annual membership, such as unlimited
rides, cost savings, and convenience.

Winter Incentives: Offer special winter promotions to encourage casual
riders to become members. Emphasize the value of having a membership
even during off-peak seasons with benefits like free or discounted
rides.

\subsubsection{Leverage Popular Start and End
Stations}\label{leverage-popular-start-and-end-stations}

Tourist Spots: Since casual riders frequently use bikes at tourist
attractions, place promotional materials and membership sign-up kiosks
at these locations. Offer special deals for tourists who sign up for
annual memberships.

Commuter Hubs: Target marketing efforts at popular start and end
stations for annual members, such as business districts and transit
hubs. Highlight the convenience and cost-effectiveness of using bikes
for daily commutes.

\subsubsection{Highlight Cost Savings and
Convenience}\label{highlight-cost-savings-and-convenience}

Cost Comparison: Create clear, compelling comparisons showing the cost
savings of annual memberships versus pay-per-ride options. Use real data
to demonstrate how much casual riders could save by switching to an
annual membership.

Convenience: Emphasize the convenience of annual memberships, such as
not having to worry about payment each time they ride and having access
to bikes anytime.

\subsubsection{Seasonal Promotions and
Discounts}\label{seasonal-promotions-and-discounts}

Weekend Specials: Since casual riders are more active on weekends, run
special promotions and discounts on annual memberships during these
times. Offer limited-time discounts to create a sense of urgency.

Referral Programs: Implement referral programs where current members can
earn rewards for bringing in new annual members. This can leverage the
existing member base to attract more casual riders.

\subsubsection{Personalized Marketing}\label{personalized-marketing}

Targeted Emails: Use data analytics to identify frequent casual riders
and send them personalized emails highlighting the benefits of annual
memberships. Include special offers and testimonials from current
members.

Social Media Campaigns: Run targeted social media ads focusing on casual
riders, showcasing the benefits of annual memberships through engaging
content, videos, and user stories.

\subsubsection{Improved User Experience}\label{improved-user-experience}

App Integration: Enhance the mobile app experience by making it easy for
casual riders to upgrade to annual memberships. Include features like
ride history, cost savings calculators, and easy sign-up processes.

Customer Support: Provide excellent customer support to address any
questions or concerns casual riders might have about switching to an
annual membership. Offer live chat support and detailed FAQs.

\subsubsection{Conclusion}\label{conclusion}

By focusing on these strategies, the marketing team can effectively
convert casual riders to annual members, leveraging data-driven insights
to tailor their approach. This will not only increase membership but
also enhance user satisfaction and loyalty.

\end{document}
